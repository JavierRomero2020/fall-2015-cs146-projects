\documentclass[12pt, oneside, letter]{article}

\usepackage{amsmath, mathtools, amsthm}
\usepackage{amssymb}
\usepackage{tikz}
\usepackage{pgfplots}
\usepackage[paperwidth=8.5in, paperheight=11in, margin=1.0in]{geometry}
\usepackage{algorithmicx}
\usepackage{algorithm}
\usepackage{algpseudocode}

\pagenumbering{gobble}

%Code to make fancy square root
\usepackage{letltxmacro}
\makeatletter
\let\oldr@@t\r@@t
\def\r@@t#1#2{%
\setbox0=\hbox{$\oldr@@t#1{#2\,}$}\dimen0=\ht0
\advance\dimen0-0.2\ht0
\setbox2=\hbox{\vrule height\ht0 depth -\dimen0}%
{\box0\lower0.4pt\box2}}
\LetLtxMacro{\oldsqrt}{\sqrt}
\renewcommand*{\sqrt}[2][\ ]{\oldsqrt[#1]{#2}}
\makeatother

\LetLtxMacro{\oldfrac}{\frac}
\renewcommand*{\frac}[2]{\oldfrac{\displaystyle #1}{\displaystyle #2}}
%End code

% Random commands for maths
\newcommand{\eq}[1]{$\displaystyle #1$} % For inline maths
\newcommand{\equ}[1]{\begin{equation*}#1\end{equation*}} % For mathematical displays
\newcommand{\equn}[1]{\begin{equation}#1\end{equation}} % For mathematical displays with numbers
\newcommand{\equs}[1]{\begin{equation*}\begin{split}#1\end{split}\end{equation*}} % For mathematical displays with splitting
\newcommand{\vecb}[1]{\mathbf{#1}} % Bold vectors
\newcommand{\vecu}[1]{\hat{\mathbf{#1}}} % Bold unanit vectors

\newcommand{\env}[2]{
    \begin{#1}
        #2
    \end{#1}
}

%\linespread{2.0}
%\setlength{\jot}{0.2in}

\begin{document}
    
    \flushright
    Jeremy Asuncion \\
    September 9th, 2015 \\
    CS146 \\
    Project 1 Write Up Questions

    \flushleft
    \env{enumerate} {
        \item
            I found my previous knowledge on stacks and linked lists very helpful for this project.

        \item
            I ran the Reverse program on the audio file for both list and array stack implementations, and for
            both double and generic versions.

        \item
            I didn't use any other Java classes for my solution.

        \item
            The underlying array starts at \eq{n = 10}, and doubles whenever the array fills up. It follows
            the following pattern: \eq{10, 20, 40, 80, 160, 320,...} This sequence has the following pattern:
            \equ {
                a_n = 10 \cdot 2^n,
            }
            where \eq{a_n} is the size of the array after \eq{n} doublings.

            Assuming the array is size \eq{n = 1,000,000} for \eq{1,000,000} lines, then there should be about
            \equ{
                \lg\left( \frac{1,000,000}{10} \right) \approx 17
            }
            doublings. For one billion lines, it's about 27 doublings, and for one trillion lines, it's about
            37 doublings.

        \item
            I'd implement a stack using the push and pop operations of the FIFO Queue class.

        \item
            \env{algorithm} {
                \caption{Push and Pop Operations}

                \env{algorithmic} {
                    \Procedure{Push}{Item x}
                        \State queue.Enqueue(x)
                    \EndProcedure

                    \Procedure{Pop}{} 
                        \If{queue.IsEmpty()}
                            \State Throw New EmptyStackException()
                        \EndIf

                        \State Return queue.Dequeue()
                    \EndProcedure
                }
            }

        \item
            While using an instance of a FIFO queue would ease the burden of handling a stack,
            it could be slightly more inefficient to keep an instance for each stack instance.
            Rather, the better option would be to either implement a stack by hand, or to
            extend the FIFO queue class and create push and pop methods 
            that wraps the enqueue and dequeue methods. Even then,
            that would add method call overhead, but it shouldn't be that bad.
    }    
    
\end{document}
